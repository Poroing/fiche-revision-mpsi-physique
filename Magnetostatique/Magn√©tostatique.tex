\title{Magnétostatique}
\author{MPSI}
\begin{document}
\maketitle
\tableofcontents
\chapter{Magnétostatique}
\section{Loi de Biot et Savart}
Une particule chargée au repos crée un champ électrique $\overrightarrow{E}$. Si une charge est mise en mouvement, on peut lui associer un courant électrique.\\
Ce courant I crée un champ magnétique $\overrightarrow{B}$.\\
La relation entre I et $\overrightarrow{B}$ est donnée par la relation de Biot et Savart
\begin{theo}
Considérons un circuit filliforme, parcouru par un courant I constant. Soit P un point quelconque du circuit, au voisinage duquel on défini une longeur élementaire $\overrightarrow{dl}$ orienté dans le sens de I.\\
On obtient la formule de Biot et Savart : 
$$d\overrightarrow{B}(M) = \dfrac{\mu_0.I.\overrightarrow{dl}\wedge\overrightarrow{PM}}{4\pi.PM^2}$$
Cette loi est vérifié en régime indépendant du temps, dans le vide.
\end{theo}
\begin{prop}
De cette formule, de par la présence du produit vectorielle, on en déduit l'existance d'une antisymétrie entre la cause I et l'effet $\overrightarrow{B}$
\end{prop}
\begin{prop}
Le champ résultant en M est donnée par : 
$$\overrightarrow{B}(M) = \int_{fil} d\overrightarrow{B}(M)$$ 
\end{prop}
\chapter{Flux du champ magnétique}
\section{Flux conservatif}
\begin{prop}
À l'aide de la loi de Biot et Savard, on montre de $\overrightarrow{B}(M)$ est à flux conservatif. Ceci signifie que l'intégrale de $\overrightarrow{B}(M)$ sur une surface fermé est toujours nulle : 
$$\oint\oint_{M \in S} \overrightarrow{B}(M).\overrightarrow{dS} = 0$$
\end{prop}
\section{Variation du champ magnétique dans un tube de champ}
Les lignes de champs de $\overrightarrow{B}$(M) sont partout tangents à $\overrightarrow{B}$(M).\\
On obtient donc que quand $\overrightarrow{B}(M)$ est uniforme, ses lignes de champs sont parralèle.
\begin{de}
En retirant un "certain" nombre de lignes de champs, on défini un tube de champs.
\end{de}
\begin{prop}
En explicitant la conservation du flux de $\overrightarrow{B}$(M) sur le tube de champs, on obtient que : 
$$B_1dS_1 = B_2dS_2$$
La conservation du flux de $\overrightarrow{B}(M)$ implique donc que $\overrightarrow{B}(M)$ est plus intense dans les zones d'étrangelement du tube de champs.
\end{prop}
\begin{prop}
Par application du théorème de Gauss dans un tube de champs : 
$$\iint_{M \in S} \overrightarrow{E}(M)\overrightarrow{dS} = \dfrac{Q_{int~ de~ S}}{\varepsilon_0}$$
\end{prop}
\chapter{Circulation du champ magnétique sur un contour fermé - Théorème d'Ampère}
\begin{de}
On défini le courant enlacé, notée $I_{enlace}$, comme la somme algébrique des courants traversant la surface S. Ces courants sont compté positivement si ils traversent S par sa face Sud, et négativement si ils traversent par sa face Nord.
\end{de}

\begin{enon}
La circulation du champ magnétique sur un contour fermé $\varphi$ orienté est égale au produit de la perméabilité du vide par le courant enlacé par le contour :
$$C = \oint_{M \in \varphi}\overrightarrow{B}(M).\overrightarrow{dl} = \mu_0.I_{enlace}$$
\end{enon}
\section{Symétrie}
\begin{prop}
Une bonne utilisation du théorème d'Ampère repose sur l'observation attentive de la topologie de $\overrightarrow{B}$, exactement comme pour le théorème de Gauss.\\
Par application du principe de Curie, on en déduit que tous plans de symétrie de translation ou de rotation pour I est perpendiculaire à $\overrightarrow{B}(M)$ et que les variables naturelles du courant sont les variables naturelles du champ magnétique 
\end{prop}

