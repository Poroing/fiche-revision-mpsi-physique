
%%% Relecture : 20 novembre 2008
\chapter{Optique géométrique}

\section{Généralités}

Considérons une onde lumineuse.\
Soit $\lambda$ sa longeur d'onde, T sa periode spatiale.\
Notons n l'indice du milieu, avec n($\lambda$) dans le cas d'un milieu dispersif.

\begin{itemize}
 \item[$\rightarrow$] Soit c, la célérité de la lumière : $$\lambda = c.T~ (\mbox{Dans le vide})$$
 \item[$\rightarrow$] Soit v, la vitesse du rayon dans un milieu d'indice n : $$v = \dfrac{c}{n}$$
\end{itemize}

Dans un prisme, une source polychromatique (ex : lumière blanche) est décomposée, avec sa composante bleue qui est plus déviée que sa composante rouge.\
Dans un milieu homogène, la lumière se propage en ligne droite

\subsection{Vocabulaire}

Plus un milieu possède un indice important, plus on dit de celui-ci qu'il est réfringent.\
Un milieu est dit homogène si ses propriétés physiques sont invarentes

\section{Relations de Snell-Descartes}

\begin{de}
Considérons deux mileux (homogènes) d'indices différents ($n_1$ et $n_2$).\
On appelle dioptre, la surface de séparation entre ces deux mileux.\
On appelle plan d'incidence, notée $\pi$, le plan défini par le rayon incident, et la normale en I, le point où le rayon touche le dioptre.\
Sur le dioptre, le rayon incident subit une réflexion et une réfraction. Ces rayons vérifient les lois de Snell-Descartes.
\end{de}

\subsection{Lois de Snell-Descartes}

Soit $n_1,n_2$ les indices respectifs des milieux 1 et 2.\
Soit $i_1,i_2$ les angles respectivement formés avec la normale par le rayon incident et le rayon réfracté.
\begin{itemize}
 \item[$\rightarrow$] Le rayon réflechi et le rayon réfracté sont dans le plan d'incidence $\pi$
 \item[$\rightarrow$] Le rayon réflechi est symétrique au rayon incident par rapport à la normale
 \item[$\rightarrow$] Loi des sinus : $n_1sin(i_1)=n_2sin(i_2)$
\item[$\rightarrow$] Loi de retour inverse : Si un rayon va du point A au point B, il utilisera le même itinéraire pour faire le chemin inverse.
\end{itemize}

\subsection{Angle limite et reflexion totale}

Considérons un rayon évoluant d'un milieu plus réfringent vers un milieu moins réfringent ($n_2$<$n_1$).\
À l'aide de la loi des sinus, on observe l'existence d'un angle limite, notée $i_L$ :
$$sin(i_L) = \dfrac{n_2}{n_1}$$
Pour tous rayons ayant un angle d'incidence i > $i_L$, il y a réflexion totale dans le milieu le plus réfrigent.

\section{Déviation}

\begin{de}
Considérons un système optique, noté S.O.\
On appelle déviation d'un rayon lumineux par un S.O. l'angle entre le rayon incident et le rayon émergent.\
Si l'on convient d'une orientiation des angles, la déviation, notée D, peut être algébrique.
\end{de}

\section{Vision d'image, conditions de Gauss}

\begin{de}
En optique géométrique, une image résulte de l'intersection de rayons ou de supports de rayons issus d'un même point objet.
\end{de}

\begin{de}
On appelle axe optique l'axe de symétrique du S.O. orienté dans la direction du rayon incident.
\end{de}

\subsection{Réel et Virtuel}

On défini des objets réels et virtuels, respectivement des images réelles et virtuelles, de la façon suivantes
\begin{center}
% use packages: array
\begin{tabular}{|l|l|l|}
\hline
 & Réel & Virtuel \\\hline
Objet (Rayon incident) & Vient de l'objet & Semble converger vers l'objet \\\hline
Image (Rayon emergent)& Converge vers l'image & Semble provenir de l'image \\\hline
\end{tabular}
\end{center}

\begin{de}
On dit d'un système optique qu'il est stigmatique pour un couple de points (A,A'), si tous rayons issus de A passe par A' après avoir traversé le S.O.\
Le seul système optique rigoureusement stigmatique est le mirroir plan.
\end{de} 

\begin{de}
On dit d'un système optique qu'il est aplanétique si l'image A'B' de l'objet AB, normal à l'axe optique, et également normal à l'axe optique
\end{de} 

\section{Conditions de Gauss}

En général, un système optique peut satisfaire un stigmatisme approché dans un S.O. en se placant dans les conditions de Gauss :
\begin{itemize}
 \item[$\rightarrow$] Le rayon lumineux est peu incliné par rapport à l'axe optique
 \item[$\rightarrow$] Le rayon lumineux est peu éloigné de l'axe optique
\end{itemize}
Hors de ces conditions, on risque d'observer une image floue, accompagnée d'abérations chromatiques et géométriques.

\section{Lentille mince dans les conditions de Gauss}

Une lentille mince résulte de l'association de deux dioptres sphériques, généralement l'un en Flint et l'autre en Crown.\
On dit d'une lentille qu'elle est mince quand son épaisseur "e" est négligable devant chacun des rayons de courbure des dioptres sphériques.\

\subsection{Lentille convergente}

Ces lentilles vérifient une série de propriétés : 
\begin{itemize}
 \item[$\rightarrow$] Tous rayons incidents parallèles à l'axe optique d'une lentille convergente passe par un point particulier appelé foyer image de la lentille, noté F'.
 \item[$\rightarrow$] Une lentille est symétrique dans son action sur la lumière. A tous foyers images F' on associe un foyer objet F par symétrie par rapport à O.
 \item[$\rightarrow$] Tous rayon incident passant par F, le foyer objet, donne un rayon parallèle à l'axe optique après la traversée d'une lentille convergente.
 \item[$\rightarrow$] Tous rayons incidents passant par O, le centre de la lentille, n'est pas déviés.
\end{itemize}

\subsection{Lentille divergente}

Ces lentilles vérifient une série de propriétés : 
\begin{itemize}
  \item[$\rightarrow$] Tous rayons incidents dont le support passe par le foyer F d'une lentille divergente donne un rayon parallèle à l'axe optique après passage du S.O.
 \item[$\rightarrow$] Une lentille est symétrique dans son action sur la lumière. A tous foyers images F' on associe un foyer objet F par symétrie par rapport à O.
  \item[$\rightarrow$] Tous rayons incidents parallèles à l'axe optique d'une lentille divergente donne un rayon divergent donc le support passe par le foyer image F'
 \item[$\rightarrow$] Tous rayons incidents passant par O, le centre de la lentille, n'est pas déviés.
\end{itemize}

\subsection{Distance focale et vergence}

\begin{de}
On appelle distance focale image, la distance entre le centre optique O d'une lentille et le foyer image F' :
$$f' = \overline{OF'}$$
Pour les lentilles convergentes, f'>0, pour les lentilles divergentes, f'<0
\end{de}

\begin{de}
On appelle vergence d'une lentille l'inverse de la distance focale :
$$v = \dfrac{1}{f'}$$
L'unité de la vergence est la dioptrie, notée $\delta$
\end{de}

\subsection{Relation de conjugaison}

Dans le cas d'une lentille mince, les relation de conjugaison est obtenue grâce à la relation de Thalès :
\begin{itemize}
  \item Relation de Descarts:
    $$\dfrac{1}{\overline{OA'}} - \dfrac{1}{\overline{OA}} = \dfrac{1}{f'}$$
  \item Relation de Newton:
    $$\overline{FA} \times \overline{FA'} = - f'^2$$
\end{itemize}

\subsection{Plan focal et foyer secondaire}

\begin{de}
Le plan focal est défini comme la perpendiculaire à l'axe optique passant par un foyer.\
Pour une lentille mince, on défini le plan focal objet, noté $\pi$, et le plan focal image, noté $\pi'$. 
\end{de}

On associe à ce plan focal un ensemble de propriétés : 
\begin{itemize}
 \item[$\rightarrow$] Un faiseau de lumière parallèle, incliné par rapport à l'axe optique d'une lentille convergente focalise en un point sur le plan focal objet, appelé foyer secondaire.\
Sa position est donnée par un rayon incident passant par O.
\item[$\rightarrow$] Un faiseau de lumière parallèle incliné par rapport à l'axe optique d'une lentille divergente donne un faisceau divergent dont le support focalise en un point sur le plan focal image, appelé foyer secondaire.\
Sa position est donnée par un rayon incident passant par O.
\end{itemize}
Grâce à ces propriétés, on peut étudier tous types de rayons incidents. On dit que l'objet est à l'infini ou que l'image d'un objet est à l'infini si respectivement les rayons incidents ou émergents sont parallèles à l'axe optique

