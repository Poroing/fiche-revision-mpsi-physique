\chapter{Théorèmes généraux}
Tous ces théorèmes ne sont applicable que dans le cas d'un dipole linéaire, c'est à dire un dipole donc la caractéristique est linéaire.
\section{Diviseur de tension}
Considérons l'association de n conducteurs ohmiques en série entre deux points A et B.\\
Soit $U_j$ la tension parcourant le $j^{eme}$ conduteur ohmique, de résistance $R_j$ :
$$U_j = \left( \dfrac{R_j}{\sum_{k=1}^n R_k}\right)U_{AB} $$
\section{Diviseur de courant}
Considérons deux résitances en dérivation.\\
Soit i le courant entrant, $i_1$ le courant traversant la résistance $R_1$.\\
On obtient :
$$i_1 = \left( \dfrac{R_2}{R_1+R_2}\right)i$$
\section{Théorème de Kennely}
Considérons un montage en étoile contenant les résistances $R_1,R_2,R_3$.\\
On obtient le montage équivalent, avec un seul noeud, contenant les résistances $r_1,r_2,r_3$, à l'aide des relations :
$$r_1 = \dfrac{R_2.R_3}{\sum_i R_i}$$
avec $r_1$ du coté de A, on a donc au numérateur $R_2,R_3$, qui sont eux aussi du coté de A.
\section{Théorème de Millman}
Considérons un circuit composé de plusieurs dipôles : sources de courants, sources de tensions, résistances.
Soit A et B deux points du circuits.\\
Le théorème de Millman permet de déterminer la tension $U_{AB}$ de la façon suivante :
$$U_{AB} = \dfrac{\sum_{j=1}^m \dfrac{U_{jB}}{R_j} + \sum_{k=1}^M \pm i_k}{\sum_{j=1}^n \dfrac{1}{R_j}}$$
