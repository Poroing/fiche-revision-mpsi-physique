\documentclass[a4paper,12pt]{report}

\usepackage[utf8x]{inputenc}			  % Utilisation du UTF8
\usepackage{textcomp}				  % Accents dans les titres
\usepackage [ french ] {babel}                    % Titres en français
\usepackage [T1] {fontenc} 			  % Correspondance clavier -> document
\usepackage[Lenny]{fncychap}                      % Beau Chapitre
\usepackage{dsfont}                    	  % Pour afficher N,Z,D,Q,R,C
\usepackage{fancyhdr}                             % Entete et pied de pages
\usepackage [outerbars] {changebar}               % Positionnement barre en marge externe
\usepackage{amsmath}				  % Utilisation de la librairie de Maths
%\usepackage{amsfont}				  % Utilisation des polices de Maths
\usepackage{cite}                                 % Citations de la bibliographie
\usepackage{openbib}                              % Gestion avancée de Bibtex
\usepackage{enumerate}				  % Permet d'utiliser la fonction énumerate
\usepackage{dsfont}				  % Utilisation des polices Dsfont
\usepackage{ae}					  % Rend le PDF plus lisible

\newtheorem{de}{Définition}
\newtheorem{theo}{Théorème}
\newtheorem{prop}{Propriété}

\title{Élements Historiques}
\author{MPSI}
\begin{document}
\maketitle
\tableofcontents
\chapter{Optique}
\begin{itemize}
 \item[$\rightarrow$] $III^{eme}$ Siècle av. JC : Euclide
 \item[$\rightarrow$] $III^{eme}$ Siècle av. JC : Archimède
 \item[$\rightarrow$] $II^{eme}$ Siècle ap. JC : Ptolémée
 \item[$\rightarrow$] $X^{eme}$ Siècle : Ibm Al Haytham
 \item[$\rightarrow$] 1609 : Lunette de Galilée
 \item[$\rightarrow$] 1690 : Traité de la Lumière par Huygens
 \item[$\rightarrow$] 1665 - 1666 : Étude de la composition de la lumière blanche par Newton
 \item[$\rightarrow$] 1672 : $1^{er}$ Télescope, celui de Newton
 \item[$\rightarrow$] 1803 : Éxperience des trous de Young
 \item[$\rightarrow$] $XIX^{eme}$ Siècle : Éxperience et la lentille de Fresnel, Modèle ondulatoire de la lumière
\item[$\rightarrow$] 1872: Lumière décrite comme une onde électromagnétique par Maxwell
\item[$\rightarrow$] 1886 : Confirmation des experiences de Maxwell par Hertz
\item[$\rightarrow$] 1889 : Effet photoélectrique par Hertz ( Modèle corpusculaire)
\item[$\rightarrow$] 1905 : Modèle onde-corpuscule d'Albert Einstein 
\item[$\rightarrow$] 1923 : Dualité onde-corpuscule proposé pour la lumière généralisé à tout les corpuscules en mouvement, par Louis de Broglie
\end{itemize}
\chapter{Mécanique}
\begin{itemize}
 \item[$\rightarrow$] -585 : Thales de Millet, -550, son éleve, Anaximandre
 \item[$\rightarrow$] -$V^{eme}$ Siècle : Pythagore défini la terre comme sphérique 
 \item[$\rightarrow$] -$IV^{eme}$ Siècle : Platon, éleve de Pythagore introduit l'atome
 \item[$\rightarrow$] -384,-322 : Aristote introduit sa pense : \begin{center}
                                                                 ``Tout objet possède une place naturelle qu'il occupe sauf si on l'en empèche''
                                                                \end{center}
 Il introduit aussi le géocentrisme
 \item[$\rightarrow$] Ptolémée : Son oeuvre etait stocké à Alexandrie. Euclide en fût le principale rapporteur. 
 \item[$\rightarrow$] -$III^{eme}$ Siècle : Erathostène défini la circonference de la terre
 \item[$\rightarrow$] 600 : Hypathie, Massacre ``réalisé pour les sciences'', époque de l'obscurantisme en Europe.
 \item[$\rightarrow$] 800 - 1200 : Apogée des sciences Arabes.
 \item[$\rightarrow$] 1200 - 1300 : Création de l'université, récupération et traduction des ouvrage arabes, et dans le meme temps, création de l'inquisition.
 \item[$\rightarrow$] 1543 : Mort de Copernic
 \item[$\rightarrow$] 1571 - 1630 : J. Kepler enoncé ses lois éponyme.
 \item[$\rightarrow$] 1564 - 1642 : Galilée et l'héliocentrisme, Messager des Étoiles (1610). Il est jugée en 1633 par les religieux.
 \item[$\rightarrow$] 1642 - 1727 : Newton, il effectue ces ``grandes vacances'' en 1665-1666
\item[$\rightarrow$] Toricelli, assistant de Galilée à la fin de sa vie.
\item[$\rightarrow$] Halley, introduit la gravitation et publie les travaux de Newton (1687)
\item[$\rightarrow$] Uranus découvert en 1781, Neptune en 1886
\end{itemize}

\end{document}
