\chapter{Equation differentielle croisée}
\section{Exemple}
Considérons le système suivant : 
\[\left\{\begin{array}{l}
  \mathring{x}\mathring{} = 0  \\
  \mathring{y}\mathring{} = \mathring{z}.\omega\\
  \mathring{z}\mathring{} = \dfrac{e.E}{m} - \omega.\mathring{y}\\
  \end{array}\right.\]
Posons : $\underline{u} = y + i.z$.
On obtient donc, en faisant $\mathring{y}\mathring{} + i.\mathring{z}\mathring{}$ :
$$\mathring{y}\mathring{} + i.\mathring{z}\mathring{} = \omega.(\mathring{z}-i.\mathring{y}) + i.\dfrac{e.E}{m}$$
D'ou :
$$\underline{\mathring{u}\mathring{}} = -i.\omega.\underline{\mathring{u}} + i.\dfrac{e.E}{m}$$
A partir de cette équation, on la résoud comme une équation différentielle habituelle.\\
Puis par identification de la partie réelle et de la partie imaginaire, on obtient les expressions de $\mathring{y}\mathring{}$ et $\mathring{z}\mathring{}$

