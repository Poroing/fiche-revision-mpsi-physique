\documentclass[a4paper,12pt,oneside]{report}

\usepackage[utf8x]{inputenc}			  % Utilisation du UTF8
\usepackage{textcomp}				  % Accents dans les titres
\usepackage [ french ] {babel}                    % Titres en français
\usepackage [T1] {fontenc} 			  % Correspondance clavier -> document
\usepackage[Lenny]{fncychap}                      % Beau Chapitre
\usepackage{dsfont}                    	  % Pour afficher N,Z,D,Q,R,C
\usepackage{fancyhdr}                             % Entete et pied de pages
\usepackage [outerbars] {changebar}               % Positionnement barre en marge externe
\usepackage{amsmath}				  % Utilisation de la librairie de Maths
%\usepackage{amsfont}				  % Utilisation des polices de Maths
\usepackage{cite}                                 % Citations de la bibliographie
\usepackage{openbib}                              % Gestion avancée de Bibtex
\usepackage{enumerate}				  % Permet d'utiliser la fonction énumerate
\usepackage{dsfont}				  % Utilisation des polices Dsfont
\usepackage{ae}					  % Rend le PDF plus lisible

\newtheorem{de}{Définition}
\newtheorem{theo}{Théorème}
\newtheorem{prop}{Propriété}


\title{Portrait de phase}
\author{MPSI}
\begin{document}
\maketitle
\chapter{Définitions}
\section{Espace des phases}
Considérons un oscillateur mécanique.
\begin{de}
On appele espace de phase de l'oscillateur un système de coordonnées dans lequel on place x(t) en abscisse et $\mathring{x}(t)$ en ordonnées
\end{de}
\subsection{Trajectoire de phase}
\begin{de}
On appele trajectoire de phase, la courbe décrite par l'ensemble des points de phases (x(t),$\mathring{x}(t)$), au cours du temps.
\end{de}
\subsection{Protrait de phase}
\begin{de}
On appele portrait de phase, l'ensemble des trajectoires de phases décrites par un système, pour différentes conditions initiales.
\end{de}

\end{document}
