\chapter{Charge électrique}
\begin{de}
La matière est électriquement neutre. Cette neutralité résulte d'une fine compensation entre les charges positives et les charges négatives dans la matière.\\
La charge électrique est une grandeur extensive, conservative (non crée et non détruite) et quantifié. (Ceci n'est plus vérifié dans le cadre d'une étude au niveau nucléaire)
\end{de}
\begin{de}
Une charge est dites ponctuelle quand on peut concentrer la totalité de la charge en un point. On la note q.
\end{de}
\section{Distribution de charge}
On distingue trois types de distributions de charges :
\begin{itemize}
 \item[$\rightarrow$] Distribution volumique
 \item[$\rightarrow$] Distribution surfacique
 \item[$\rightarrow$] Distribution linéique
\end{itemize}
\subsection{Distribution volumique}
Soit une distribution volumique de charges, pour toute la charge $Q_T$, dans un volume v.\\
Soit M un point appartenant au voisinage du quel on défini un volume dv(M). Ce volume dv(M) porte la charge dq(M)
\begin{de}
On appelle densité volumique de charge, défini au voisinage de M, la grandeur suivante :
$$\rho(m) = \dfrac{dq(M)}{dv(M)}$$
Son unité est le C.$m^{-3}$
\end{de}
\begin{prop}
Une densite volumique est considérer uniforme quand : 
$$\rho(M) = \rho_0 = cte$$
\end{prop}
\begin{prop}
La charge totale portée par v est donnée en fonction de $\rho(M)$ par :
$$Q_T = \iiint_V dq(M) = \iiint_V \rho(M).dv(M)$$
Si la distribution est uniforme : 
$$Q_T = \rho_0.v$$
\end{prop}
\subsection{Distribution surfacique et linéaire}
Si respectivement l'epaisseur ou la section est négligable devant la dimension du corps considéré, on peut effectuer les approximations suivantes : 
\begin{itemize}
 \item[$\rightarrow$] Dans le cadre d'une distribution surfacique, dq(M) $\simeq \sigma(M).dS(M)$
 \item[$\rightarrow$] Dans le cadre d'une distribution linéique, dq(M) $\simeq \lambda(M).dL(M)$
\end{itemize}
On obtient donc dans le cas d'une distribution uniforme : 
\begin{itemize}
 \item[$\rightarrow$] Dans le cadre d'une distribution surfacique $Q_T = \sigma_0.S$
 \item[$\rightarrow$] Dans le cadre d'une distribution linéique, $Q_T = \lambda_0.L$
\end{itemize}
\section{Loi de Coulomb}
\begin{de}
Soient $q_1,q_2$ deux charges ponctuelle plongées dans le vide. La force d'interaction électrostatique exercée par $q_1$ sur $q_2$ est donnée par la loi de Coulomb : 
$$\overrightarrow{F}_{1 \rightarrow 2} = \dfrac{q_1.q_2}{4\pi \varepsilon_0 r^2} \overrightarrow{u_r} = -\overrightarrow{F}_{2 \rightarrow 1}$$
Cette force est en $\dfrac{1}{r^2}$. On dit qu'elle est newtonienne. C'est une force conservative.
\end{de}
\subsection{Principe de superposition}
Considérons une charge ponctuelle M'(q') de charge q' soumise à l'action de N charges ponctuelles $q_i$.\\
La force ressenti par M'(q') est : 
$$\overrightarrow{F}_{res} = \sum_i^N \dfrac{q_i.q'}{4\pi\varepsilon_0r_i^2}\overrightarrow{u}_i$$
avec : $\overrightarrow{u}_i = \dfrac{\overrightarrow{M_iM'}}{||\overrightarrow{M_iM'}||}$.\\
Ceci constitue le principe de superposition. On postule qu'il y a linéarité entre la cause et les effets.\\
On peut généraliser ceci pour une distribution volumique, surfacique ou linéique.
\section{Champs électrostatique}
L'intéret de définir un champs est qu'on se libère du sujet d'étude. Un champs est une propriété intrinsèque de l'espace.
\subsection{Champs crée par une charge ponctuelle}
\begin{de}
Soient deux charges ponctuelles, M(q) de charge q, et M'(q') de charge q'. La force exercé par q sur q' est donnée par la loi de Coulomb.\\
Le champs électrique en M' est donnée par : 
$$\dfrac{\overrightarrow{F}_{M \rightarrow M'}}{q'} = \overrightarrow{E}(M') = \dfrac{q}{4\pi \varepsilon_0 r^2} \overrightarrow{u_r} $$
\end{de}
\subsection{Champs crée par une distribution de charge}
\subsubsection{Distribution volumique de charge}
Soit P un point quelconque d'une distribution volumique de charge, au voisinage duquel on défini un volume élementaire dv(p), contenant la charge dq(p) = $\rho(P).dv(P)$.\\
Soit M un point quelconque de l'espace.\\
En partant de l'expression du champs crée par un charge ponctuelle (Méthode de calcul directe), on obtient : 
$$\overrightarrow{E}(M) = \dfrac{1}{4\pi\varepsilon_0} \iiint_V \dfrac{\rho(P).\overrightarrow{PM}.dv(P)}{(PM)^3}$$
\subsubsection{Distribution surfacique de charge}
On obtient par un raisonnement analogue :
$$\overrightarrow{E}(M) = \dfrac{1}{4\pi\varepsilon_0} \iint_S \dfrac{\sigma(P).dS(P)}{(PM)^3}$$
\subsubsection{Distribution linéique de charge}
De même :
$$\overrightarrow{E}(M) = \dfrac{1}{4\pi\varepsilon_0} \int_L \dfrac{\lambda(P).dL(P)}{(PM)^3}$$
\subsection{Champs crée par un disque uniformement chargé en surface avec la densité $\sigma_0$}
Par application des formules précedentes, et en s'appuyant sur les symétries du problèmes, on obtient, avec $\alpha$ l'angle entre l'axe de symétrie et le sommet du disque, et $\overrightarrow{u}$ vecteur directeur de l'axe de symétrie : 
$$\overrightarrow{E}(M) = \dfrac{\sigma_0}{2.\varepsilon_0}(1 - cos(\alpha))\overrightarrow{u}$$
\subsection{Champs crée par un fil de longeur infini uniformement chargé avec la densité $\lambda_0$}
Par application, on obtient : 
$$\overrightarrow{E}(M) = \dfrac{\lambda_0}{2\pi\varepsilon_0.OM} \overrightarrow{u}$$
