\chapter{Mouvement d'une particule chargée}
\section{Force de Lorentz}
\subsection{Champs électrique, champs magnétique}
\begin{de}
A chaque fois qu'on détecte une différence de potentiel entre deux points de l'espace, on introduit une grandeur vectorielle, appelé champs électrique, notée $\overrightarrow{E}$, d'unité Volt/mètre.
\end{de}
\begin{de}
Un aimant ou un circuit parcouru par un courant peux avoir une action sur une boussole. On modélise cette action par un champs magnétique, notée $\overrightarrow{B}$, d'unité le Tesla.
\end{de}
\subsection{Force de Lorentz}
\begin{de}
Soit M(m,q) une particule de masse m et de charge q observée dans le référentiel R galiléen.\\
Si M est plongée dans la zone d'action d'un champs électromagnétique ($\overrightarrow{E},\overrightarrow{B}$) avec un vitesse $\overrightarrow{V}(M)_R$, elle subit la force de Lorentz :
$$\overrightarrow{F}_l = q(\overrightarrow{E}+\overrightarrow{V}(M)_R\wedge\overrightarrow{B})$$
\end{de}
\subsection{Invarience relativiste}
\begin{prop}
On peut définir un référentiel R' dans lequel $\overrightarrow{E'}=\overrightarrow{0}$.\\
On observe donc que le champs $\overrightarrow{E}$ et $\overrightarrow{B}$ ne suffisent pas pris à part, on prend donc en compte pour l'étude d'un système le couple : $(\overrightarrow{E},\overrightarrow{B})$.
\end{prop}
\begin{prop}
En général, au cours de l'étude d'une particule élémentaire, on néglige l'influence de son poids.
\end{prop}
\section{Mouvement d'une particule chargé dans un champs électrique}
\subsection{Trajectoire}
Soit $\overrightarrow{E}$ un champs électrique de norme E.
Soit M(m,$e^-$) un électron assimilé à un point materiel de masse m et de charge $e^-$ observé dans R galiléen.\\
On obtient, par application du P.F.D, l'équation horaire : 
$$y(x) = \dfrac{e.E}{m.v_0^2}.x^2$$
La trajectoire est donc une parabole.
\subsection{Focalisation}
Considérons un faisceau de particules chargées passant dans la zone d'action de : $$\overrightarrow{E}=E.\overrightarrow{i}$$ 
avec un vecteur vitesse 
$$\overrightarrow{v}_0 = v_0.cos(\alpha).\overrightarrow{i} + v_0.sin(\alpha).\overrightarrow{j}$$ on obtient : 
$$x(y) = \dfrac{-E_C}{2.m.(v_0.sin(\alpha))^2}.y(t)^2 + \dfrac{y(t)}{tan(\alpha)}$$
On obtient la porté maximale, pour x(y) = 0 et y $\neq$ 0 : 
$$y_p = \dfrac{m.v_0^2}{e.E}.sin(2\alpha)$$
La portée est maximale pour $\alpha = \dfrac{\pi}{4}$. Pour des angles proche de $\alpha$, le faisceau focalise toujours en $y_p$.
\subsection{Canon à électrons}
\begin{de}
Pour émettre des électrons, on chauffe un filamment et on impose aux électrons de passer par des petites ouvertures.\\
Les électrons sont arrachés au filamment. Soit B le point d'entré de la phase d'accélération, et A le point de sortie. On obtient : 
$$dEc_{B\rightarrow A} = q(V_B-V_A)$$
En négligant la diffraction en A ($\lambda_{dB} = \dfrac{h}{p} \ll d$, avec d diamètre de l'ouverture), on obtient : 
$$\overrightarrow{v}_0 = \sqrt{\dfrac{2.e.U_{AB}}{m}}.\overrightarrow{i}$$
\end{de}
\section{Étude du mouvement d'une particule chargée dans un champs magnétique}
\begin{prop}
Par application du P.F.D, on montre qu'un champs magnétique ne peux pas modifier la norme de la vitesse, mais il peux modifier la trajectoire de la particule.
\end{prop}
\begin{prop}
Si $v_0$ est perpendiculaire à B, la trajectoire de M(m,q), point materiel de masse m et de charge q, est circulaire uniforme.
\end{prop}
\begin{prop}
Si $v_0$ est quelconque, la trajectoire de M(m,q), point materiel de masse m et de charge q, est hélicoidale uniforme.
\end{prop}
\subsection{Mouvement d'une particule dans un champs électromagnétique}
\begin{prop}
Soit M(m,q) une particule de masse m et de charge q, placée à l'origine d'un référentiel galiléen, sans vitesse initiale.\\
Par application du P.F.D. et de la méthode de résolution des équations couplées, on obtient que la trajectoire de M est une cycloïde.
\end{prop}
\begin{prop}
Sachant que : 
$$\overrightarrow{E'} = \overrightarrow{E} + \overrightarrow{u}\wedge\overrightarrow{B}$$
avec $\overrightarrow{E'}$ le champs électrique dans un référentiel R' et $\overrightarrow{u}$ un vecteur quelconque.\\
En posant $\overrightarrow{u} = \dfrac{E}{B}.\overrightarrow{j}$, on obtient que dans le référentielle R' en translation uniforme par rapport à R de vecteur $\overrightarrow{u}$.
\end{prop}
