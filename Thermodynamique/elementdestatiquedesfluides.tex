\documentclass[a4paper,12 pt,oneside]{report}     % Type de document
\usepackage[utf8x]{inputenc}			  % Utilisation du UTF8
\usepackage{textcomp}				  % Accents dans les titres
\usepackage [ french ] {babel}                    % Titres en français
\usepackage [T1] {fontenc} 			  % Correspondance clavier -> document
\usepackage[Lenny]{fncychap}                      % Beau Chapitre
\usepackage{dsfont}                    	  % Pour afficher N,Z,D,Q,R,C
\usepackage{fancyhdr}                             % Entete et pied de pages
\usepackage [outerbars] {changebar}               % Positionnement barre en marge externe
\usepackage{amsmath}				  % Utilisation de la librairie de Maths
%\usepackage{amsfont}				  % Utilisation des polices de Maths
\usepackage{cite}                                 % Citations de la bibliographie
\usepackage{openbib}                              % Gestion avancée de Bibtex
\usepackage{enumerate}				  % Permet d'utiliser la fonction énumerate
\usepackage{dsfont}				  % Utilisation des polices Dsfont
\usepackage{ae}					  % Rend le PDF plus lisible
\usepackage[pdftex]{graphicx}
\usepackage{graphics}

\newtheorem{de}{Définition}
\newtheorem{theo}{Théorème}

%opening
\title{Elements de statique des fluides}
\author{MPSI}
\begin{document}

\maketitle
\tableofcontents

\chapter{Éléments de statique des fluides}
\begin{de}
Un fluide est un milieu continu, déformable. On considère dans cette définition les liquides et les gaz. Un fluide est considéré comme parfait si son coefficiant de viscosité est nul.
\end{de}
\section{Pression et force pressante}
On appelle pression exercé par le fluide en M, le scalaire défini par :
$$d\overrightarrow{F_{p}} = p(M).dS(M)\overrightarrow{n}$$
Système d'unité :
\begin{itemize}
 \item 1 bar = $10^{5} Pa$
 \item 1 tor, (De Torricelli) = pression exercé par 1mm de Hg ( Mercure)
 \item 1 atm = 1,013 bar
\end{itemize}
\section{Force Résultante exercé par un fluide}
Considérons un solide immergé dans un fluide.
La force pressante résultante exercé par un fluide sur ce solide, est la somme des forces $d\overrightarrow{F}(M)$ :
$$\overrightarrow{F_{res}} = \iint_{Surface} d\overrightarrow{F_{(M)}} = - \iint_{Surface} p(M).dS\overrightarrow{n_{e}}$$
avec $\overrightarrow{n_e}$ vecteur unitaire, orienté vers l'extérieur, normal à dS(M)
\section{Particule de Fluide}
Considérons un fluide quelconque. Soit M un point quelconque dans le fluide.
Soit $\rho$(M), masse volumique dans le fluide, au voisinage de M
Soit $d_M$, masse d'un volume élementaire au voisinage de M, $dV_M$
$$d_M = \rho(M).dV_M$$
\section{Force pressante exercé par un fluide sur une particule de fluide}
Considérons une particule de fluide cubique définie au voisinage de M(x,y,z).
Soit $d\overrightarrow{F_{res}}$ la force exercé par le fluide sur la particule de fluide.
$$d\overrightarrow{F_{res}} = -\overrightarrow{grad}(p).dV_m$$
\section{Définition d'un gradiant}
\begin{de}
Soit p la pression du fluide, avec p = p(x,y,z)
 $$(\dfrac{\partial p}{\partial x})_{y,z}\overrightarrow{i} + (\dfrac{\partial p}{\partial y})_{x,z}\overrightarrow{j} + (\dfrac{\partial p}{\partial z})_{x,y}\overrightarrow{k} = \overrightarrow{grad}(p) $$
On le défini aussi à l'aide de la relation : 
$$dp(M) = \overrightarrow{grad}(p).\overrightarrow{dl}$$
\end{de}
\section{Loi fondamental de la statique, dans un référentiel galiléen}
\begin{de}
Considérons un fluide au repos dans le référentiel R galiléen
Considérons une particule de fluide définie au voisinage d'un point M.
La masse de cette particule de fluide est :
$$d_m = \rho(M).dV_m$$
Considérons que la particule de fluide est au repos. Grâce au P.F.D., on obtient : 
$$\overrightarrow{grad}(p) = \rho(M).\overrightarrow{g}$$
Ceci constitue la loi fondamental de la statique des fluides dans R.
\end{de}
\section{Théorème de Pascal}
\begin{de}
Un fluide est dit incompressible si :
$$\rho(M)=\rho_0=cte$$
\end{de}
Considérons un fluide incompressible.
Par application de la loi fondamental de la statique, on obtient : 
$$\overrightarrow{grad}(p) = \rho(M).\overrightarrow{g}$$
Sachant que la variation de pression est donnée par : 
$$dp(M) = \overrightarrow{grad}(p).\overrightarrow{dl}$$
En explicitant, on obtient la relation de Pascal :
$$p(M) = p_0 + \rho_0.g.z$$
avec ici, z > 0
\begin{theo}
"Un fluide incompressible transmet intégralement les variations de pressions"
\end{theo}
\section{Fluide compressible assimilable à un gaz parfait}
Considérons un gaz parfait isotherme.
En utilisant la loi fondamentale de la statique des fluides et la définition du gradiant, on obtient : 
$$p(z) = p_0e^{\dfrac{-Mgz}{Rt_0}}$$
\section{Loi fondamentale de la statique des fluides dans un référentiel non galiléen}
Considérons une particule M(dM) de masse dM au repos dans un référentiel non galiléen
On ajoute deux forces dans la somme des forces du PFD : \\
\begin{enumerate}
 \item La force d'entraînement $d\overrightarrow{F_{ie}} = -d_m.\overrightarrow{a_e}$ 
 \item La force de coriolis $d\overrightarrow{F_{ic}} = -d_m.\overrightarrow{a_c}=\overrightarrow{0}$ \\
\end{enumerate}
Ceci nous conduit à la loi fondamentale de la statique dans un référentiel non galiléen :
$$\overrightarrow{grad}(p) = \rho(M).(\overrightarrow{g}-\overrightarrow{a_e})$$
Si le repère est non galiléen, et qu'il est en : \\
\begin{enumerate}
 \item Translation rectiligne par rapport à un référentiel galiléen, alors :\\
	$$\overrightarrow{a_e}=a_0.\overrightarrow{i}$$
  avec $a_0$ accélération de ce repère par rapport à celui galiléen
 \item Rotation uniforme autour d'un axe fixe du référentiel galiléen :
$$\overrightarrow{ac} = -\omega^{2}\overrightarrow{R}$$
avec $\overrightarrow{R}$ la distance entre l'axe de rotation et le point \\
\end{enumerate}
Les surfaces isobare sont perpendiculaire à $\overrightarrow{g}$ dans un référentiel galiléen,et perpendiculaire à $\overrightarrow{g}-\overrightarrow{a_e}$
\section{Force pressante et poussé d'Archimède}
\begin{de}
On appelle poussé d'Archimède la force définie par : 
$$\overrightarrow{F_{res}} = -\int\int_{surface}p(m).dS.\overrightarrow{n_e}$$
\end{de}
\begin{theo}
 "Tous corps immergé dans un ou plusieurs fluide subit une force pressante résultante opposé au poids du fluide déplacé."
Cette force est notée $\overrightarrow{\pi}$
\end{theo}
\end{document}
