\documentclass[a4paper,12pt,oneside]{report}

\usepackage[utf8x]{inputenc}			  % Utilisation du UTF8
\usepackage{textcomp}				  % Accents dans les titres
\usepackage [ french ] {babel}                    % Titres en français
\usepackage [T1] {fontenc} 			  % Correspondance clavier -> document
\usepackage[Lenny]{fncychap}                      % Beau Chapitre
\usepackage{dsfont}                    	  % Pour afficher N,Z,D,Q,R,C
\usepackage{fancyhdr}                             % Entete et pied de pages
\usepackage [outerbars] {changebar}               % Positionnement barre en marge externe
\usepackage{amsmath}				  % Utilisation de la librairie de Maths
%\usepackage{amsfont}				  % Utilisation des polices de Maths
\usepackage{cite}                                 % Citations de la bibliographie
\usepackage{openbib}                              % Gestion avancée de Bibtex
\usepackage{enumerate}				  % Permet d'utiliser la fonction énumerate
\usepackage{dsfont}				  % Utilisation des polices Dsfont
\usepackage{ae}					  % Rend le PDF plus lisible

\newtheorem{de}{Définition}
\newtheorem{theo}{Théorème}
\newtheorem{prop}{Propriété}


\title{Résumé}
\author{MPSI}
\begin{document}
\section{Définitions}
\begin{itemize}
 \item[$\rightarrow$] $dp(M) = \overrightarrow{grad}(p).\overrightarrow{dl}$
 \item[$\rightarrow$] $\overrightarrow{grad}(p) = \rho(M).\overrightarrow{g}$
\end{itemize}
\section{Introduction à la thermodynamique}
\begin{itemize}
 \item[$\rightarrow$] $U_{GPM} = \dfrac{3}{2}.n.R.T$
 \item[$\rightarrow$] $U_{GPD}(T=basse) = \dfrac{3}{2}.n.R.T$
 \item[$\rightarrow$] $U_{GPD}(T=amb) = \dfrac{5}{2}.n.R.T$
 \item[$\rightarrow$] $U_{GPD}(T=haute) = \dfrac{7}{2}.n.R.T$
\item[$\rightarrow$] $\Delta u = n.C_{v,mol}.\Delta T$
\item[$\rightarrow$] $\Delta H = n.C_{p,mol}.\Delta T$
\item[$\rightarrow$] $\alpha = \dfrac{1}{V}\left(\dfrac{\partial V}{\partial T} \right)_p $
\item[$\rightarrow$]$\xi_T = -\dfrac{1}{V}\left(\dfrac{\partial V}{\partial p} \right)_T$
\item[$\rightarrow$]$\beta = \dfrac{1}{p}\left(\dfrac{\partial p}{\partial T} \right)_V $
\end{itemize}
\section{Premier principe}
\begin{itemize}
 \item[$\rightarrow$] $\delta w = -p_{ext}(M).dV$ 
 \item[$\rightarrow$] Évolution isochore : $\Delta u = Q$
 \item[$\rightarrow$] Évolution isobare : $\Delta H = Q$
\item[$\rightarrow$] Évolution isentropique : $p(t).V^{\gamma} = cte$
 \item[$\rightarrow$] $H = u + p.V$
 \item[$\rightarrow$] $C_{v,mol}=\dfrac{R}{\gamma - 1}$
 \item[$\rightarrow$] $C_{p,mol}=\dfrac{\gamma.R}{\gamma - 1}$
 \item[$\rightarrow$] Pour une phase condensé : $du = dH = C.dT$
\item[$\rightarrow$] $dS = \dfrac{du}{T}+\dfrac{p.dV}{T}$ ou $dS = \dfrac{dH}{T}-\dfrac{V.dp}{T}$ 
\end{itemize}
\section{Machine Thermique}
\begin{itemize}
 \item[$\rightarrow$] Un cycle est moteur si il est décrit dans le sens horaire dans un diagramme de Clapeyron 
 \item[$\rightarrow$] $\nu$ = $\dfrac{\mbox{Énergie~ utile}}{\mbox{Énergie~ dépensé}}$
\end{itemize}
\section{Transition de phase}
\begin{itemize}
 \item[$\rightarrow$] Dans une évolution isotherme : $\Delta H = ml = L= Q$
 \item[$\rightarrow$] Dans une évolution isotherme et isobare : $\Delta S = \dfrac{ml}{T}$
\end{itemize}
\end{document}
