%% Relecture : 20 novembre 2008 

\chapter{Mecanisme réactionnels}
\section{Processus élementaire}
\begin{de}
Un processus élémentaire décrit la réaction entre quelques entitées chimique, à l'échelle atomique ou moléculaire. Le nombre de réactifs mis en jeu dans un processus élémentaire consitute sa molécularité. Le nombre d'atomes ou de molécules mis en jeu dans un processus élémentaire est toujours entier.
\end{de}
\section{Loi de Van't Hoff}
\begin{loi}
L'ordre partiel $\alpha_i$ du réactif $A_i$ dans un processus élémentaire s'identifie à son coefficiant stochiométrique
\end{loi}
\section{Mécanisme réactionnel}
\begin{de}
On appelle mécanisme réactionnel l'ensemble des processus élémentaires nécessaire pour décrire la cinétique d'une réaction complexe. Les entitées chimique présentes dans le mécanique réactionnel sont des intermédiaires réactionnels, qui peuvent ou non apparaitre dans l'équation bilan.
\end{de}
\section{Principe de Bodenstein}
\begin{de}
Soit X un intermédiaire réactionnel. Si X n'apparait pas dans l'équation bilan, on peut faire l'hypothèse que : $$\dfrac{d[X(t)]}{dt} = 0$$
\end{de}
\section{Hypothèse de l'étape cinétiquement limitante}
\begin{de}
Si une constante de vitesse est très faible devant toutes les autres, c'est cette constante qui condition la vitesse globale de la réaction.
\end{de}
\section{Réaction en chaine}
Une réaction en chaine décrit trois étapes :
\begin{enumerate}[1-]
 \item L'initiation : $A \rightarrow B$
 \item La boucle : $B \rightarrow C + D$ ; $C \rightarrow B$. Elle peut se réaliser un très grand nombre de fois.
 \item La rupture : $B \rightarrow E$
\end{enumerate}
Bien souvent, dans les réactions en chaine, on peut appliquer l'hypothèse de l'étape cinétiquement limitant, conscient que la boucle s'effectue un très grands nombres de fois devant les autres phases.