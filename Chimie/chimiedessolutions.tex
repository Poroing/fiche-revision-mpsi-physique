%%% Relecture : 20 novembre 2008 

\chapter{La chimie des solutions}
\section{L'eau, molécule et solvant}
\subsection{Moment dipolaire}
\begin{de}
On défini le vecteur dipolaire $\overrightarrow{P}$ par:
$$\overrightarrow{P} = q\overrightarrow{NP}$$
avec :
\begin{itemize}
 \item[$\rightarrow$] N : Barycentre des charges négatives
 \item[$\rightarrow$] P : Barycentre des charges positives
 \item[$\rightarrow$] q : Charge associée au dipole
\end{itemize}
Son unité est le Debye, notée B.
$$1D = \dfrac{1}{3}.10^{-29}C.m$$
A température ambiante, nous avons : p($H_2O$) = 1,85D
\end{de}
\subsection{Force d'interaction}
En partant de la loi de Coulomb : 
$$F_{vide} = \dfrac{q_1.q_2}{4.\pi.\varepsilon_0.r^2}$$
On obtient la force reliant deux charges dans l'eau :
$$F_{eau} = \dfrac{F_{vide}}{\varepsilon_R}$$
avec $\varepsilon_R$ : Permitivité relative à l'eau, égale à 80 à température ambiante. Ceci nous renseigne sur le caractère polarisant de la molécule d'eau.
\section{Réactions chimiques}
Considérons la réaction suivant :
$$\nu_1.A_1 + \nu_2.A_2 + .... \rightarrow \nu_1'.A_1' + \nu_2'.A_2' + .....$$
\subsection{Avancement d'une réaction chimique}
\begin{de}
L'avancement à l'instant t est donné par :
$$\xi(t) = \dfrac{n_i - n_i(t)}{\nu_i} = \dfrac{n_i'(t) - n_i'}{\nu_i'} $$
L'avancement ne dépend pas des entités chimiques mise en jeu. C'est une grandeur caractéristique de la réaction.
\end{de}
\subsection{Équilibre chimique}
\begin{de} 
On considère que la réaction ci-dessus est à l'équilibre chimique quand la vitesse de réaction, définie par :
$$v = \dfrac{1}{\nu'_1}.\dfrac{d[A'_i(t)]}{dt} = -\dfrac{1}{\nu_1}.\dfrac{d[A_i(t)]}{dt}$$
est nulle.
\end{de}
\subsection{Relation de Guldberg et Waages}
\begin{de}
On appelle quotient de cette réaction, noté Q(t), le produit de l'activité des produits, affecté de leurs coefficiants stochiométriques, sur le produit des réactifs affecté de leurs coefficiants stochiométriques :
$$Q(t) = \dfrac{a(A'_1)^{\nu'_1}.a(A'_2)^{\nu'_2}....}{a(A_1)^{\nu_1}.a(A_1)^{\nu_1}...}$$ 
\end{de}
On défini l'activité des différentes entitées chimiques de la façon suivante :
\begin{itemize}
 \item[$\rightarrow$] Si X est le solvant (eau) : $$a(X) = 1$$
 \item[$\rightarrow$] Si X est un soluté non missible en solution, ou un liquide non missible : $$a(X) = 1$$
 \item[$\rightarrow$] Si X est un soluté missible en solution, alors, avec $C_0$ : Concentration de référence, souvent 1 mol.$l^{-1}$. $$a(X) = \dfrac{[X(t)]}{C_0}$$
 \item[$\rightarrow$] Si X est un gaz, alors, avec $p_0$ : Pression de référence, souvent 1 bar. $$a(X) = \dfrac{p(X(t))}{p_0}$$
\end{itemize}
L'activité est donc une grandeur sans dimensions.\\
Cette relation permet d'établir qu'à l'équilibre chimique, le quotient de la réaction est une constante qui ne dépend que de la température. Cette constante est notée K(T) : $$K(T)=Q(t_e)=cte$$
\section{Réactions acido-basique}
\begin{de}
Selon la définition de Bronsted, un acide est une entitée susceptible de céder un ou plusieurs protons
\end{de}
\subsection{Constante d'acidité}
Considérons le couple acido-basique AH/$A^-$. Ce couple réagit avec l'eau selon la réaction :
$$AH + H_20 = A^- + H_30^+$$
Par application de la relation de Guldberg et Waages, on défini la constante de réaction :
$$K(T) = \dfrac{[A^-].[H_3O^+]}{[AH]} = cte$$
Cette constante de réaction est appelé constante d'acidité, notée $K_A(T)$.
\subsection{Constante de basicité}
En utilisant le caractere amphotère (ampholyte)\footnote{Entitée chimique qui peut être, selon les cas, une base ou un acide} de l'eau, on defini la constante de basicité comme la constante de la réaction suivante :
$$A^- + H_20 = AH + OH^-$$
On la note $K_B(T)$ :
$$K_B(T) = \dfrac{[AH].[HO^-]}{[A^-]}$$
\subsection{Produit ionique de l'eau}
Conscient que l'eau est un composé amphotère, on défini le produit ionique de l'eau comme la constante de réaction de la réaction suivante :
$$2H_2O = H_30^+ + HO^-$$
On le note $K_e(T)$
$$K_e(T) = [H_3O^+].[HO^-]$$
\subsection{Relations entre $K_A$ et $K_B$ - $pK_A$ et $pK_B$}
On obtient la relation suivante :
$$K_A=\dfrac{K_e}{K_B}$$
On note pX = -log(X), avec X une grandeur sans dimension. On obtient une seconde relation :
$$pK_A + pK_B = pK_e$$
\subsection{Diagramme de prédominance}
Pour tracer un diagramme de prédominance, on isole $[H_3O^+]$ de la constante d'acidité $K_A$. Puis on passe au pH = p([$H_3O^+$]). On obtient une formule de la forme :
$$pH = pK_A + log\left( \dfrac{[A^-]}{[AH]}\right) $$
On obtient donc la frontière entre les deux domaines de prédominance : pH = $pK_A$. On détermine les pH de prédominance à partir de cette formule.\\
On dit qu'une entitée est majoriaire si elle est au moins 10 fois plus concentrée que son entitée conjugée.
\subsection{Force d'un acide ou d'une base - Nivellement par l'eau}
Plus le $pK_A$ d'un couple est faible, plus l'acide est fort. Respectivement, plus le $pK_A$ d'un couple est élevé, plus la base est forte.\\
Les acides les plus fort ne sont pas observable dans l'eau, on dit que ces acides sont nivellé par l'eau.
\section{Réactions de complexation}
On peut modéliser les réactions de complexation par une réaction du type :
$$M + nL = ML_n$$
avec :
\begin{itemize}
 \item[$\rightarrow$] M : Cation métalique (ex : $Fe^{3+}$)
 \item[$\rightarrow$] L : Ligand (ex: Molécule avec un atome possèdant un doublé non liant, un anion)
 \item[$\rightarrow$] M$L_n$ : L'ion complexe
\end{itemize}
\subsection{Constante de formation et de dissociation d'un ion complexe}
On appelle constante de formation d'un ion complexe, notée $K_F$, la constante de la réaction de complexation donnée par la relation de Guldberg et Waages.\\
Respectivement, la constante de dissociation d'un ion complexe, notée $K_D$, est la constante de la réaction de dissocitation donnée par la relation de Guldberg et Waages :
$$K_D = \dfrac{1}{K_F}$$
Quand on procède à la création d'un ion complexe en de multiples étapes, on remarque que la constante de réaction finale est égale au produit des constantes des réactions intermédiaires.
\subsection{Diagramme de prédominance}
Pour tracer un diagramme de prédominance, on isole $[HO^-]$ de la constante de formation $K_F$. Puis on passe au pOH = p([$OH^-$]). On obtient une formule de la forme :
$$pOH = pK_D + log\left( \dfrac{[M]}{[ML_n]}\right) $$
\section{Réactions de précipitation}
\begin{de}
Les réactions de précipitation sont des cas particuliers de réactions de complexation dans lesquels le produit de la réaction est électriquement neutre, c'est donc un précipité.
\end{de}
La réaction de précipitation est caractérisée par une constante de dissociation appelé produit de solubilité du soluté, notée $K_s$, à l'équilibre chimique (donc quand la solution est saturée).\\
Par application de la relation de Gouldberg et Waages :
$$K_s = [M].[L]^n$$ 
\subsection{Diagramme de prédominance}
On obtient une équation de la forme : 
$$p(L) = pK_s + log(M)$$
On travaille à partir de cette équation.
\subsection{Solubilité}
\begin{de}
La solubilité d'un soluté, notée s, correspond au nombres de moles de soluté que l'on peut dissoudre par litre de solution.\\
Son unité est mol.$l^{-1}$
\end{de}
