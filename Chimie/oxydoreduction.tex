%%% Relecture : 20 novembre 2008
\chapter{Réaction d'oxydo-réduction}
\begin{de}
Une réaction d'oxydo-réduction est une réaction durant laquelle il y a échange d'électrons.\\
L'équation type de réaction est :
$$Ox + ne^- \rightarrow Red$$
\end{de}
\section{Nombres d'oxydation}
Les nombres d'oxydation permettent d'équilibrer une équation. Ils sont toujours écrit à l'aide de chiffre romaine. Ils sont régit par les règles suivantes :\\
\begin{itemize}
 \item[$\rightarrow$] Dans une entitée monoatomique, le nombre d'oxydation est la charge de l'entitée.\\
 \item[$\rightarrow$] Dans une entitée polyatomique, la somme des nombres d'oxydation des différents élements est égale à la charge de l'entitée.\\
 \item[$\rightarrow$] En géneral : n.o(H) = I ; n.o(O) = -II\\
\end{itemize}
Les nombres d'oxydations permettent de savoir si une entitée X est oxydée ou réduite, et de connaitre le nombre d'électrons échangés :\\
\begin{itemize}
 \item[$\rightarrow$] Si $\Delta n.o(X) > 0 $ : Il y a oxydation de X\\
 \item[$\rightarrow$] Si $\Delta n.o(X) < 0 $ : Il y a réduction de X\\
\item[$\rightarrow$] |$\Delta n.o(X)$| = Nombres d'électrons échangés\\
\end{itemize}
\subsection{Plan d'équilibrage d'une réaction avec les nombres d'oxydation}
\begin{itemize}
 \item[$\rightarrow$] On calcule les nombres d'oxydations des differentes entitées mise en jeu.\\
 \item[$\rightarrow$] On équilibre les $\Delta$n.o sachant qu'il faut que : $$\sum_i \nu_i \Delta n.o (X_i) = 0$$
 \item[$\rightarrow$] On vérifie la conservation des charges, si besoin on ajoute des $H^+$ en milieu acide, des $OH^-$ en milieu basique\\
 \item[$\rightarrow$] On vérifie la conservation de la matière, si besoin on ajoute des $H_2O$. \\
 \item[$\rightarrow$] On vérifie si la réaction est bien équilibrée.\\
\end{itemize}
\section{Formule de Nernst}
\subsection{Vocabulaire}
\begin{de}
Les notions d'anonde et de cathode sont défini par la polarité de la pile. Nous retiendrons que la réduction s'effectue à la cathode.
\end{de}
\subsection{Potentiel d'électrode}
\begin{de}
On appelle potentiel d'électrode le potentiel d'une électrode de mesure par rapport à l'électrode de réference, à savoir l'électrode standard à hydrogène (E.S.H)
\end{de}
\subsection{Enoncé}
\begin{de}
Considérons le couple oxydant-réducteur dont la demi-équation électronique est :
$$\alpha Ox+ne^- = \beta.Red$$
La formule de Nernst établie que le potentiel d'électrode relatif à ce couple a pour expression :
$$E(Ox/Red) = E^0(Ox/Red) + \dfrac{R.T}{n.F}.ln\left( \dfrac{a^{\alpha}(g(Ox))}{a^{\beta}(g(Red))}\right) $$
On obtient l'expression usuelle, à la température de 25°C :
$$E(Ox/Red) = E^0(Ox/Red) + \dfrac{0.06}{n}.log\left( \dfrac{a^{\alpha}(g(Ox))}{a^{\beta}(g(Red))}\right) $$
avec $g(Ox)$ ke groupe oxydant et g(Red) le groupe réducteur
\end{de}
A partir de cette expression, on peut établir le diagramme de prédominance.
\section{Réaction d'oxydo-réduction à l'équilibre chimique}
Condérons la réaction :
$$A.Ox_1+B.Re_2 = A.Red_1+B.Ox_2$$
A l'équilibre, nous avons : 
$$E(Ox_1/Red_1) = E(Ox_2/Red_2)$$
A partir de la relation de Guldberg et Waages, on obtient l'expression de la constante de réaction :
$$K(t) = 10^{\frac{n}{0.06}(E^0(Ox)-E^0(Red))}$$
