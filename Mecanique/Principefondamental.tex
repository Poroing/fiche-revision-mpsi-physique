\chapter{Principe fondamental de la dynamique}

\section{Masse et quantité de mouvement}

\begin{de}
La masse inerte, notée M.I, est un scalaire positif traduisant la répugnance d'un corps au mouvement.
C'est une grandeur extensive. On identifie masse inerte et masse gravitationnelle (Grâce aux travaux d'Albert Einstein).
\end{de}

\begin{de}
Soit M un point matériel de masse m (inerte), observé dans un référentiel R.\
La quantité de mouvement de M dans R est défini comme le produit de sa masse par son vecteur vitesse dans R.
$$\vec{P}_{M \in R} = m.\vec{v}_{M \in R}$$
\end{de}

\section{Interactions et forces}

\begin{de}
 On dit que deux systèmes sont en interaction quand une modification sur l'un entraine une modification sur l'autre
\end{de}

On distingue 4 forces fondamentales :
\begin{itemize}
 \item[$\rightarrow$] \textbf{\textit{Nucléaire forte}} : Cette interaction est une interaction de courte porte qui assure la cohésion du noyau.
 \item[$\rightarrow$] \textbf{\textit{Nucléaire faible}} : Cette interaction est une interaction de très courte portée. Elle apparait dans la désintégration $\beta$
 \item[$\rightarrow$]\textit{\textbf{\'Electro-magnétique}} : La matière est électriquement neutre, et cette électro-neutralité résulte d'une fine compensation entre les charges positives et négatives. L'interaction électro-magnétique est responsable des phénomènes atomique et moléculaire.
 \item[$\rightarrow$] \textit{\textbf{Gravitationnelle}} : Cette interaction est responsable du comportement des planètes et des galaxies. Elle est dû à une interaction entre masse gravitationnelle.
\end{itemize}

\section{Forces}

\subsection{\'Electro-magnétique}

Soit M, de masse m et de charge q, un point matériel, observé dans un référentiel galiléen R et plongé dans la zone d'action d'un champ électromagnétique ($\vec{E},\vec{B}$), avec $\vec{E}$ champ électrique et $\vec{B}$ champ magnétique.\
M est soumis dans R à la force de Lorentz :
$$\vec{F_L} = q.\vec{E} + q.(\vec{V_{M \in R}}\wedge\vec{B})$$
Si M est au repos dans R :
$$\vec{F_L} = q.\vec{E}$$
Si le champ électrique est crée par une charge ponctuelle q' immobile dans R, alors on obtient la loi de Coulomb : 
$$\vec{F_L} = q.\vec{E} = \dfrac{q.q'}{4\pi\varepsilon_0r^2}.\vec{e_r}$$

\subsection{Gravitationnelle}

Soit $M_1$ de masse $m_1$, soit $M_2$ de masse $m_2$ deux points matériels observé dans le référentiel R galiléen.\
L'interaction entre $M_1$ et $M_2$ dans le cadre de la mécanique classique est décrite par la force :
$$\vec{F_{1 \mapsto 2}} = -G.\dfrac{m_1.m_2}{r^2}\vec{e_r} = -\vec{F_{2 \mapsto 1}}$$
Ceci constitue la quatrième loi de Newton.

\subsection{De contact}

À l'opposé des forces précédentes, qui sont des forces à distance, les forces de contact ne s'appuie que sur l'expérience, elle ne répondent pas à des fondements théorique (Loi phénoménologique).
\begin{itemize}
 \item[$\rightarrow$] Force de frottement fluide : $\vec{f}=-\alpha \vec{v_{M \in R}}$
 \item[$\rightarrow$] Force de frottement solide : $R_t = \beta R_n = \beta mg$
\end{itemize}

\section{Principe fondamental de la dynamique}

\subsection{Référentiel galiléen}

\begin{de}
On dit d'un référentiel qu'il est galiléen si, quand M est isolé, il vérifie la première loi de Newton (Le principe d'inertie) : $$\left(\dfrac{d\vec{p_{M \in R}}}{dt}\right)_R = \vec{0}$$
En réalité, on détermine si un référentiel est galiléen par l'expérience, en vérifiant qu'il vérifie les lois de Newton
\end{de}

Un repère galiléen vérifie les propriétés suivantes :
\begin{itemize}
 \item[$\rightarrow$] La quantité de mouvement de M est constant dans R quand M est isolé : $$\vec{P_{M \in R}}=m\vec{v(m)_R}$$
 \item[$\rightarrow$] Tout référentiel en translation rectiligne uniforme par rapport à un référentiel galiléen est un référentiel galiléen.
\end{itemize}

\subsection{\'Enoncé du principe fondamental de la dynamique}

Soit M(m) un point matériel observé dans un référentiel R galiléen, et soumis à la force $\sum \vec{F_{ext}}$, la somme des forces extérieures à M.\
Le principe fondamental de la dynamique postule que :
$$\sum \vec{F_{ext}} = m.\vec{a(M)_R}$$
Ce principe est aussi connu sous le nom de seconde loi de Newton
\subsubsection{Conséquences}
Ce principe entraine, entre autres, les conséquences suivantes :
\begin{itemize}
\item[$\rightarrow$] Invariance galiléenne : Le bilan des forces extérieures est le même dans tous référentiels galiléens.
\item[$\rightarrow$] Principe d'équivalence : On obtient l'équivalence entre la masse inerte et la masse gravitationnelle.
\end{itemize}
