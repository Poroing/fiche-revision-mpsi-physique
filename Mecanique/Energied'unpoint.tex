\documentclass[a4paper,12pt,oneside]{report}

\usepackage[utf8x]{inputenc}			  % Utilisation du UTF8
\usepackage{textcomp}				  % Accents dans les titres
\usepackage [ french ] {babel}                    % Titres en français
\usepackage [T1] {fontenc} 			  % Correspondance clavier -> document
\usepackage[Lenny]{fncychap}                      % Beau Chapitre
\usepackage{dsfont}                    	          % Pour afficher N,Z,D,Q,R,C
\usepackage{fancyhdr}                             % Entete et pied de pages
\usepackage [outerbars] {changebar}               % Positionnement barre en marge externe
\usepackage{amsmath}				  % Utilisation de la librairie de Maths
%\usepackage{amsfont}				  % Utilisation des polices de Maths
\usepackage{cite}                                 % Citations de la bibliographie
\usepackage{openbib}                              % Gestion avancée de Bibtex
\usepackage{enumerate}				  % Permet d'utiliser la fonction énumerate
\usepackage{dsfont}				  % Utilisation des polices Dsfont
\usepackage{ae}					  % Rend le PDF plus lisible

\newtheorem{de}{Définition}
\newtheorem{theo}{Théorème}
\newtheorem{prop}{Propriété}

\title{Energie d'un point materiel}
\author{MPSI}
\begin{document}
\maketitle
\tableofcontents
\chapter{Puissance et travail d'une force}
\section{Puissance}
\begin{de}
Soit M(m) un point materiel observé dans un référentiel R et animé de la vitesse $\overrightarrow{V}(M)_R$.\\
Supposons que M soit soumis à l'action d'une force $\overrightarrow{F}$.\\
La puissance associé à cette force à chaque instant est défini par :
$$P = \overrightarrow{F}.\overrightarrow{V}(M)_R$$
L'unité de P est le Watt.
\end{de}
\begin{prop}
Si M(m) est soumis à un ensemble de force $\Sigma \overrightarrow{F}$, alors :
$$P = \sum_i P_i$$
avec :
$$P_i = \overrightarrow{F_i}.\overrightarrow{V}(M)_R$$
\end{prop}
\section{Travail d'une force}
\begin{de}
Le travail d'une force $\overrightarrow{F}$ entre t et t+dt dans le referentiel R est donnée par :
$$\delta\omega = P.dt$$
On obtient aussi la formule :
$$\delta\omega = \overrightarrow{F}.\overrightarrow{dl}$$
\end{de}

\end{document}
