\chapter{Système de deux points materiels}
\section{Référentiel barycentrique}
Considérons un système composé de deux points materiels $M_1(m_1)$, de masse $m_1$, et $M_2(m_2)$, de masse $m_2$. Le système est notée S = $\{M_1,M_2\}$.\\
\subsection{Barycentre}
\begin{de}
On défini le barycentre G, appelé aussi centre de gravité ou centre de masse, par la relation suivante :
$$m_1\overrightarrow{OM_1} + m_2\overrightarrow{OM_2} = (m_1+m_2) \overrightarrow{OG}$$
On obtient aussi la relation suivante : 
$$m_1\overrightarrow{GM_1} + m_2\overrightarrow{GM_2} = \overrightarrow{0}$$
\end{de}
\subsection{Référentiel barycentrique}
\begin{de}
Soit $R^*$ le référentiel barycentrique de S.\\
$R^*$ a pour origine le barycentre G du système, et est en translation par rapport à un référentiel R galiléen, ce qui ce caractérise par : 
$$\omega_{\dfrac{R^*}{R_g}} = 0$$
\end{de}
\begin{prop}
Dans le cas où S est isolé, $R^*$ est en translation rectiligine uniforme par rapport à $R_G$, c'est donc un référentiel galiléen.
\end{prop}
\subsection{Théorème de la résultante cinétique}
Par application du principe fondamental de la dynamique à $M_1$ et à $M_2$, on obtient le théorème dit de la résultante cinétique :
$$(m_1+m_2).\overrightarrow{a(G)}_R = \sum_{ext} \overrightarrow{F}$$ 
avec : 
$$\sum_{ext} \overrightarrow{F} = \sum_{ext \rightarrow 1} \overrightarrow{F} + \sum_{ext \rightarrow 2} \overrightarrow{F}$$
\section{Élements cinétique dans $R^*$ galiléen}
\subsection{Résultante cinétique}
Par définition du barycentre, on obtient :
$$\overrightarrow{p}(S)_{R^*} = \overrightarrow{0}$$
\subsection{Moment cinétique}
On obtient, dans $R^*$, l'expression suivante pour le moment cinétique, en posant $\overrightarrow{r}$=$\overrightarrow{M_2M_1}$ :
$$\overrightarrow{L}(G)_{R^*} = \overrightarrow{r}\wedge\dfrac{m_1.m_2}{m_1+m_2}.\left( \dfrac{d\overrightarrow{r}}{dt}\right)_{R^*} $$
\begin{prop}
Dans $R^*$, le moment cinétique de S est independant du point d'observation, pour vu qu'il soit fixe.
\end{prop}
\subsection{Énergie cinétique}
Soit $E_C(S)_{R^*}$ l'énergie cinétique de S dans $R^*$. On obtient : 
$$E_C(S)_{R^*} = \dfrac{1}{2}\dfrac{m_1.m_2}{m_1+m_2}\left( \dfrac{d\overrightarrow{r}}{dt}\right)_{R^*} ^2$$
\subsection{Particule fictive}
Soit M une particule fictive de masse $\mu = \dfrac{m_1.m_2}{m_1+m_2}$, appelé masse réduite et de vecteur position $\overrightarrow{r}$. $\mu$.\\
L'étude d'un système à deux points materiels peut se ramèner à celle d'une particule fictive.\\
On détermine le mouvement de $M_1$ et de $M_2$ à l'aide des relations homothétiques suivantes : 
$$\overrightarrow{GM_1} = \dfrac{m_2}{m_1+m_2}\overrightarrow{r}$$
$$\overrightarrow{GM_2} = \dfrac{-m_1}{m_1+m_2}\overrightarrow{r}$$
\section{Étude du mouvement d'un système de deux points materiels isolés, en interaction newtonienne}
Nous allons étudier le système dans le cadre d'une interaction gravitationelle, mais toutes les relations seront vérifiées pour toutes interactions newtoniennes.\\
Nous allons tout d'abord faire l'étude du mouvement de la particule fictive.
\subsection{Énergie potentielle de gravitation}
On obtient la même expression pour l'énergie potentielle que dans le cas d'un point materiel soumis à une force centrale : 
$$E_p = \dfrac{-G.m_1.m_2}{r}$$
\subsection{Énergie mécanique}
On obtient l'expression de l'énergie mécanique d'un point materiel soumis à une force conservative : 
$$E_m(S)_{R^*} = E_C(S)_{R^*} + E_p(r) = cte$$
\subsection{Nature de la trajectoire, équation du mouvement}
Par application du théorème du moment cinétique à M($\mu$), on obtient :
$$\overrightarrow{L}_{R^*} = \overrightarrow{GM}\wedge\mu \overrightarrow{v}(M)_{R^*} = \overrightarrow{cte}$$
Donc la trajectoire de M est plane. Le plan du mouvement est normal à $\overrightarrow{L}_{R^*}$, et passe par G. On en déduit que la trajectoire de $M_1$ et de $M_2$ sont plane elles aussi. Le système satisfait donc la loi des aires.
\subsubsection{Énergie mécanique}
Par application de l'énergie mécanique, et à l'aide des formules de Binet, on obtient : 
$$r(\theta) = \dfrac{p}{1 + e.cos(\theta - \theta_0)}$$
Avec :
$$p = \dfrac{\varphi}{G(m_1+m_2)}$$
On obtient la même équation par application du principe fondamentale de la dynamique, ou à l'aide de la méthode de Ronge-Lenz.
\subsection{Grandeurs caractéristiques du mouvement}
\subsubsection{Énergie mécanique et excentricité}
On obtient : 
$$E_m(M)_{R^*} = \dfrac{-G.m_1.m_2}{2a}$$
avec : 
$$a = \dfrac{p}{1-e^2}$$
\subsubsection{Relation entre periode et demi-grand axe}
On obtient, à l'aide de la vitesse aréolaire : 
$$\dfrac{T^2}{a^3} = \dfrac{4.\pi^2}{G(m_1+m_2)}$$
\subsubsection{Étude d'un système de deux points matériel}
Un système de deux points materiels isolés se ramène donc bien à l'étude d'une particule fictive dans $R^*$. 
