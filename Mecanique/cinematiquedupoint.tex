\chapter{Cinématique du point}

\section{Postulats de Newton}

\subsection{Espace et Temps}

La mécanique newtonienne repose sur les postulats spatio-temporels de Newton, à savoir :
\begin{itemize}
 \item[$\rightarrow$] L'espace est absolu, immuable, infini, euclidien, homogène et isotrope
 \item[$\rightarrow$] Le temps est absolu et uniforme
\end{itemize}

\subsection{Point matériel - Réferentiel}

\begin{de}
Un système sera assimilé à un point à partir du moment où sa position dans l'espace peut etre donnée par un triplet de coordonées.
\end{de}

\begin{de}
On parle d'un point materiel quand on concentre la totalité de la masse du système à l'isobarycentre des masses. La position de cet objet sera donc étudié depuis ce point.
\end{de}

\begin{de}
Un référentiel est un repère muni de la notion de temps. On peut donc y effectuer des mesures, et donc y réaliser une étude cinématique.
\end{de}

\section{Vitesse}

\begin{de}
Soit M un point materiel observé dans un référentiel R.\\
La position de M à l'instant t est donnée par le vecteur position $\vec{OM}$(t).\\
La vitesse de M dans R est définie par :
$$\vec{V}(M)_R = \left( \dfrac{d\vec{OM}(t)}{dt}\right)_R $$
De plus, si on considère $\vec{dl}$ un déplacement élementaire, on obtient :
$$\vec{V}(M)_R = \left( \dfrac{\vec{dl}}{dt}\right)_R $$
\end{de}

\subsubsection{Expression en coordonnées catérisienne}

$$\vec{V}(M)_R = \dot{x}\vec{e_x} + \dot{y}\vec{e_y} + \dot{z} \vec{e_z}$$

\subsubsection{Expression en coordonnées cylindrique}

$$\vec{V}(M)_R = \dot{r}\vec{e_r} + r\dot{\theta}\vec{e_{\theta}} + \dot{z} \vec{e_z}$$

\subsubsection{Expression en coordonnées polaire}

$$\vec{V}(M)_R = \dot{r}\vec{e_r} + r\dot{\theta}\vec{e_{\theta}}$$

\section{Accélération}

\begin{de}
Par définition, l'accélération de M, animé de la vitesse $\vec{V}(M)_R$ est donnée par :
$$\vec{a}(M)_R = \left( \dfrac{d\vec{V}(M)_R}{dt}\right)_R $$
\end{de}

\subsubsection{Expression en coordonnées cartérisienne}

$$\vec{a}(M)_R = \dot{x}\dot{}\vec{i} + \dot{y}\dot{}\vec{j} + \dot{z}\dot{} \vvec{e_z}$$

\subsubsection{Expression en coordonnées cylindrique}

$$\vec{a}(M)_R = (\dot{r}\dot{} - r\dot{\theta}^2)\vec{e_r} + (2\dot{r}\dot{\theta} + r\dot{\dot{\theta}})\vec{e_{\theta}} + \dot{z}\dot{}\vvec{e_z}$$

\subsubsection{Expression en coordonnées polaire}

$$\vec{a}(M)_R = a_r\vec{e_r} + a_{\theta}\vec{e_{\theta}}$$
avec : 
\begin{itemize}
 \item[$\rightarrow$] $a_r = \dot{r}\dot{} - r\dot{\theta}^2$ : C'est l'accélération radiale
 \item[$\rightarrow$] $a_{\theta} = 2\dot{r}\dot{\theta} + r\dot{\dot{\theta}}$ : C'est l'accélération orthoradiale
\end{itemize}
En coordonnée polaire, on dit qu'un système subit une accélération centrale si et seulement si l'accélération $a_{\theta}$ est nul. \\
Dans ce cas, le vecteur accélération passe par un point fixe appelé centre de force.\\
De plus, en dérivant $r^2\dot{\theta}$ par rapport au temps, on remarque que l'expression :
$$\varphi = r^2\dot{\theta} $$
est une constante.\\
On appelle $\varphi$ constante des aires
